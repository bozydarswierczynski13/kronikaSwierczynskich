Drogie moje Dzieci i drodzy Kuzyni, niniejsza ,,Historia rodziny Świerczyńskich'' ma ambicję pobudzenia Waszych duchów do zapuszczenia korzeni w głąb zajmującej historii naszych przodków, którzy, chociaż nie byli generałami, ministrami, senatorami, to właśnie dlatego, że starali się żyć uczciwie warci są naszej pamięci. Nie każdemu się to w pełni udało. Niektórym alkohol zmarnował życie, innym potężna pięść władzy lub wojna, innym jeszcze nieuleczalna choroba. Ci, którym dane było dożyć późnego wieku, są świadectwem dla nas dość czytelnym, jak Boża Opatrzność działała w nich,  by schodzili z tego świata z bogatym naręczem bezinteresownych dobrych uczynków, bowiem tylko tymi skarbami będziemy w stanie przebłagać Sprawiedliwego Boga za nasze grzechy, których zdarzyło nam się popełnić znacznie więcej, niż owych skarbów naręcze\ldots a jeszcze i to, czy dane im było pojednać się z Bogiem i czy w godziwej sprawie swoje życie położyli, jak np. nasz stryj Karol, który przepadł w kampanii wrześniowej. Nie trzeba być wielkim bohaterem, by wejść do Królestwa Niebieskiego, wystarczy żyć uczciwie i mieć miękkie serce dla głodnych, spragnionych i wszystkich innych niesłusznie pogardzanych. Spisana przeze mnie historia rodziny Świerczyńskich sięga początków wieku XIX. Oczywiście, z tamtej odległej historii pozostały tylko imiona i nazwiska zapisane w księgach chrztów, księgach ślubów oraz księgach zgonów, w których zapisane są tylko daty, świadkowie, chrzestni i ewentualnie pozycja społeczna lub zawód dorosłych uczestników aktu. Jedni byli zagrodnikami, inni chłopami pańszczyźnianymi, niektórzy bauerami\ldots a nasz prapradziadek Walenty Świerczyński był nawet wójtem. Warto znać historię swojej rodziny, gdyż ona zobowiązuje do uczciwego życia, do szlachetnych, czasem nawet bohaterskich, chwalebnych czynów.